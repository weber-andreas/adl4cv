\section{Diffusion Models}

\subsection{Denoising Diffusion Probabilistic Models (DDPMs)}

\textbf{Core Idea}
\begin{itemize}
	\item Diffusion models learn to generate data by \textbf{iterative denoising}.
	\item Two complementary processes:
	      \begin{itemize}
		      \item \textbf{Forward diffusion:} Gradually corrupt data with Gaussian noise.
		      \item \textbf{Reverse diffusion:} Learn to remove noise step-by-step to recover data.
	      \end{itemize}
	\item Grounded in nonequilibrium thermodynamics and probabilistic modeling.
\end{itemize}


\begin{center}
	\includegraphics[width=\columnwidth]{images/diffusion.jpeg}
	\label{fig:diffusion}
\end{center}


\textbf{Relevant Works:}\\
\href{https://arxiv.org/abs/1503.03585}{[Sohl-Dickstein et al., Deep Unsupervised Learning using Nonequilibrium Thermodynamics, ICML 2015]}\\
\href{https://arxiv.org/abs/2006.11239}{[Ho et al., Denoising Diffusion Probabilistic Models, NeurIPS 2020]}\\
\href{https://arxiv.org/abs/2011.13456}{[Song et al., Score-Based Generative Modeling through Stochastic Differential Equations, ICLR 2021]}



\subsubsection{Forward Diffusion Process}

\textbf{Markovian Noise Injection}
\begin{itemize}
	\item Data sample $x_0 \sim q(x)$ is progressively noised over $T$ steps.
	\item Each step depends only on the previous one:
	      \[
		      q(x_t \mid x_{t-1}) = \mathcal{N}(x_t; \mu_t = \sqrt{1-\beta_t}x_{t-1}, \Sigma_t= \beta_t I)
	      \]
	\item $\beta_t$ is a variance schedule (e.g., linear from $10^{-4}$ to $0.02$ over $T=1000$ steps).
\end{itemize}

\begin{center}
	\includegraphics[width=\columnwidth]{images/denoising_diffusion.jpeg}
	\label{fig:denoising_diffusion}
\end{center}


\textbf{Closed-Form Sampling}
Inefficient to add noise step-wise.\\
One-step prediction of a noise at a specific timestep by applying Parametrization Trick for Gaussian distributions.
\begin{itemize}
	\item Define $\alpha_t = 1 - \beta_t$, $\bar{\alpha}_t = \prod_{s=1}^t \alpha_s$.
	\item Allows direct sampling from $x_0$:
	      \[
		      x_t = \sqrt{\bar{\alpha}_t}x_0 + \sqrt{1-\bar{\alpha}_t}\epsilon, \quad \epsilon \sim \mathcal{N}(0,I)
	      \]
	\item Enables efficient training by sampling arbitrary timesteps.
\end{itemize}

\subsubsection{Reverse Diffusion Process}

\textbf{Generative Denoising}
\begin{itemize}
	\item As $t \to T = \infty $, $x_T$ approaches pure Gaussian noise.
	\item Generation starts from sampling $x_T \sim \mathcal{N}(0,I)$.
	\item Reverse process:
	      \[
		      p_\theta(x_{t-1} \mid x_t) = \mathcal{N}(x_{t-1}; \mu_\theta(x_t,t), \Sigma_t)
	      \]
	\item Since the true reverse posterior $p_\theta(x_{t-1} \mid x_t)$ is intractable, it is approximated with a parameterized model
	\item A neural network is trained to predict the parameters of this distribution, effectively learning to denoise $x_t$ step-by-step.
	\item Repeating this process from $t=T$ to $t=0$ produces a novel sample drawn from the original data distribution.
\end{itemize}

\subsubsection{Training Objective}
Optimzie negative log-likelihood of training data:
\[
	\mathcal{L} = \mathbb{E}_{x_0, \epsilon, t}\left[-\log p_\theta(x_{t-1} \mid x_t)\right]
\]
\textbf{Variational Lower Bound (VLB)}
\begin{itemize}
	\item However, direct likelihood maximization is intractable.
	\item Optimize a variational bound decomposed into timestep-wise KL terms.
	\item Most terms are either constant or simplified.
\end{itemize}

\textbf{Simplified Noise Prediction Loss}
\begin{itemize}
	\item Instead of predicting $\mu_\theta$ (image at specific timestep), predict noise $\epsilon_\theta$:
	      \[
		      \mathcal{L}_{simple} =
		      \mathbb{E}_{t,x_0,\epsilon}\left[\|\epsilon - \epsilon_\theta(x_t,t)\|^2\right]
	      \]
	\item Fixed variance schedule $\Sigma_t$ simplifies training.
	\item Leads to stable optimization and strong empirical performance.
\end{itemize}

\subsection{Denoising Diffusion Implicit Models (DDIM)}

\textbf{Motivation}
\begin{itemize}
	\item DDPMs require many sequential denoising steps ($T=1000$) for high-quality generation.
	\item DDIM accelerates sampling while maintaining quality by using a non-Markovian diffusion process.
	\item Enables deterministic generation with fewer steps.
\end{itemize}

\href{https://arxiv.org/abs/2010.02502}{[Denoising Diffusion Implicit Models, Song et al., ICLR 2021]}

\textbf{Core Idea}
\begin{itemize}
	\item DDPM sampling is inherently stochastic due to added noise at each step.
	\item DDIM proposes a non-Markovian reverse process that is deterministic.
	\item DDIM allows to skip steps during sampling, leading to less overall steps and faster generation.
\end{itemize}

\textbf{Key Advantages}
\begin{itemize}
	\item \textbf{Accelerated Sampling:} Reduce steps from 1000 to 10-50 with minimal quality loss.
	\item \textbf{Determinism:} Set $\sigma_t = 0$ for deterministic generation (same latent $z$ always produces same output).
	\item \textbf{No Retraining:} Uses existing DDPM models directly.
	\item \textbf{Semantically Meaningful Latent:} Interpolation in latent space produces smooth transitions.
\end{itemize}

\textbf{Trade-offs}
\begin{itemize}
	\item Fewer steps may reduce sample diversity if too aggressive.
	\item Stochasticity-determinism trade-off: deterministic samples have less diversity but more consistency.
\end{itemize}

\subsection{Diffusion Model Architectures}

\textbf{General Properties}
\begin{itemize}
	\item Input and output dimensions must match.
	\item Highly flexible architecture choices (U-Net, Transformer backbones)
	\item Conditioning information (time, text, camera, etc.) injected via embeddings.
\end{itemize}

\textbf{U-Net Backbone}
\begin{itemize}
	\item Most common architecture for image diffusion.
	\item Encoder-decoder with skip connections.
	\item Time-step embeddings modulate intermediate layers.
\end{itemize}

\begin{center}
	\includegraphics[width=\columnwidth]{images/diffusion_unet.jpeg}
	\label{fig:diffusion_unet}
\end{center}


\textbf{Transformer-Based Diffusion}
\begin{itemize}
	\item Replace U-Net with a Vision Transformer backbone.
	\item Patchify images and treat diffusion as sequence modeling.
	\item Strong scaling behavior with large datasets and models.
\end{itemize}


\subsection{Latent Diffusion Models (U-Net)}

\textbf{Motivation}
\begin{itemize}
	\item Pixel-space diffusion is computationally expensive.
	\item Latent diffusion operates in a compressed representation.
\end{itemize}

\begin{center}
	\includegraphics[width=\columnwidth]{images/latent_diffusion.jpeg}
	\label{fig:latent_diffusion}
\end{center}


\textbf{Two-Stage Training}
\begin{enumerate}
	\item Train an autoencoder (AE / VAE / VQ-VAE / VQ-GAN).
	\item Perform diffusion in the learned latent space.
\end{enumerate}

\begin{itemize}
	\item Reduces memory and compute cost.
	\item Enables high-resolution synthesis (e.g., Stable Diffusion).
\end{itemize}

\textbf{Findings:}
\begin{itemize}
	\item Latent diffusion for large-scale structure in images
	\item Autoencoder/GAN for fine details and textures
\end{itemize}

\begin{center}
	\includegraphics[width=\columnwidth]{images/latent_diffusion_vs_gan.jpeg}
	\label{fig:latent_diffusion_vs_gan}
\end{center}
\begin{itemize}
	\item Condition latent diffusion model on text embeddings, class labels, or other modalities.
	\item Enables powerful conditional generation (e.g., text-to-image).
	\item More guidance improves training signal and sample quality.
\end{itemize}

\begin{center}
	\includegraphics[width=\columnwidth]{images/latent_diffusion_conditioning.jpeg}
	\label{fig:latent_diffusion_conditioning}
\end{center}

\subsection{Diffusion Transformers (DiT)}
\begin{itemize}
	\item Train on image patches and treat diffusion as sequence modeling.
	\item Use transformer blocks with self-attention and cross-attention.
	\item Scales well with model size and dataset size.
	\item Achieves state-of-the-art image generation quality.
\end{itemize}

\begin{center}
	\includegraphics[width=\columnwidth]{images/diffusion_transformer.jpeg}
	\label{fig:diffusion_transformer}
\end{center}
\href{https://arxiv.org/abs/2212.09748}{Scalable Diffusion Models with Transformers, DiT, CVPR 2023}


\subsection{Conditional Diffusion}

\textbf{Conditional Generation}
\begin{itemize}
	\item Model reverse process as $p_\theta(x_{t-1} \mid x_t, y)$.
	\item Supports text-to-image, class-to-image, and other modalities.
\end{itemize}

\textbf{Classifier-Free Guidance}
\begin{itemize}
	\item Randomly drop condition during training.
	\item Intuition: takes steps towards fewer, higher-quality modes
	\item At inference, combine conditional and unconditional predictions:
	      \[
		      \hat{\epsilon} = (1+\omega)\epsilon_\theta(x_t,t,y) - \omega \epsilon_\theta(x_t,t)
	      \]
	\item Trades diversity for higher fidelity.
	\item Most prominent: ControlNet
\end{itemize}

\subsection{ControlNet}

\begin{itemize}
	\item Learn to control a pretrained model with an additional input condition $c$.
	\item Copy weights from pretrained diffusion model and add trainable "zero-conv" blocks.
	\item Zero-initialized convolutions ensure initial behavior matches pretrained model.
	\item Train only the control branch on paired data $(x,c)$.
\end{itemize}

\begin{center}
	\includegraphics[width=\columnwidth]{images/control_net.jpeg}
	\label{fig:control_net}
\end{center}

\href{https://arxiv.org/abs/2302.05543}{[Adding Conditional Control to Text-to-Image Diffusion Models, ControlNet, CVPR 2023]}\\

\textbf{Controlled Generation}
\begin{itemize}
	\item Adds spatial or structural conditions (edges, depth, poses).
	\item Uses:
	      \begin{itemize}
		      \item \textbf{Frozen pretrained diffusion model}
		      \item \textbf{Trainable control branch}
	      \end{itemize}
	\item Zero-initialized convolutions prevent early distortion.
	\item Enables training with limited additional data.
\end{itemize}

\subsection{Multi-View Diffusion Models (MVDiffusion)}

\textbf{View-Consistent Image Generation}
\begin{itemize}
	\item Generate multiple views consistent with 3D geometry.
	\item Conditioning via camera parameters (R, T) or ray-based encodings.
\end{itemize}

\begin{center}
	\includegraphics[width=\columnwidth]{images/mvdiffusion.jpeg}
	\label{fig:mvdiffusion}
\end{center}

\textbf{Representative Methods}
\begin{itemize}
	\item \textbf{Zero-1-to-3:} Finetunes image diffusion on view pairs.
	\item \textbf{SyncDreamer:} Joint denoising of multiple views.
	\item \textbf{CAT3D:} Camera conditioning using raymaps.
\end{itemize}

\subsection{Video Diffusion Models}

\textbf{Temporal Generation}
\begin{itemize}
	\item Extend diffusion to spatio-temporal domains.
	\item Key challenge: temporal consistency.
\end{itemize}

\textbf{Latent Video Diffusion}
\begin{itemize}
	\item Operate in compressed video latent space.
	\item Combine image and video pretraining.
	\item Examples: Align Your Latents, MovieGen, Sora.
\end{itemize}

\textbf{Align your Latents}
\begin{itemize}
	\item Inject temporal information into latent diffusion.
	\item Use temporal attention layers in U-Net.
	\item Achieves high-quality video generation from text prompts.
\end{itemize}

\begin{center}
	\includegraphics[width=\columnwidth]{images/align_latents.jpeg}
	\label{fig:align_latents}
\end{center}


\subsection{General Latent Design}
Either use pre-trained image-based latend diffusion models or train directly on video latents.
\begin{enumerate}
	\item Pre-trained image-based LDM: \begin{itemize}
		      \item Can leverage massive amounts of pre-training
		      \item Issues in low-level temporal consistency
	      \end{itemize}
	\item Train directly on video latents: \begin{itemize}
		      \item Better temporal consistency
		      \item Requires large-scale video datasets for training
		      \item However, no pre-trained models can be used (expensive)
	      \end{itemize}
\end{enumerate}


\subsection{Autoregressive Video Generation}
Directly trained on video latents.
\textbf{NOVA}
\begin{itemize}
	\item Generates videos frame-by-frame in latent space.
	\item Uses optical flow for motion guidance.
	\item More efficient and less error accumulation than diffusion-only approaches.
\end{itemize}


\subsection{3D-Aware Diffusion}
\textbf{Explicit 3D Supervision}
\begin{itemize}
	\item Train diffusion models with 3D ground truth.
	\item Example: DiffRF predicts radiance fields via diffusion.
\end{itemize}

\textbf{Latent 3D Gaussian Diffusion (L3DG)}
\begin{itemize}
	\item Diffusion in sparse latent space of 3D Gaussians.
	\item Enables object- and scene-level 3D generation.
	\item Scales to room-sized environments.
\end{itemize}

\subsection{Score Distillation Sampling (SDS)}

\textbf{Text-to-3D via Diffusion Priors}
\begin{itemize}
	\item Use pretrained 2D diffusion models as supervision.
	\item Optimize 3D representation by matching diffusion scores.
	\item Introduced in DreamFusion.
	\item Slow: for each gradient step we need to perform a forward pass through the NeRF rendering.
\end{itemize}

\begin{center}
	\includegraphics[width=\columnwidth]{images/dream_fusion.jpeg}
	\label{fig:dream_fusion}
\end{center}

\textbf{Follow-Up Work}
\begin{itemize}
	\item ProlificDreamer (variational SDS).
	\item DreamGaussian (Gaussian-based 3D generation).
\end{itemize}

\subsection{World and Scene Generation}
\textbf{Motivation}
\begin{itemize}
	\item Manual creation of scenes is labor-intensive and requires time and expertise.
	\item Challenges  \begin{itemize}
		      \item scale
		      \item consistency
		      \item detail
		      \item context window: 3D scene diverge from previous views after a few steps (3D caching)
	      \end{itemize}
	\item Idea: leverage powerful 2D diffusion priors to generate 3D content.
\end{itemize}

\textbf{Leveraging 2D Priors for 3D Worlds}
\begin{itemize}
	\item Iteratively lift generated images into 3D.
	\item Render-refine-repeat pipelines.
\end{itemize}

\textbf{Representative Systems}
\begin{itemize}
	\item Text2Room
	\item ControlRoom3D
	\item WonderWorld
	\item WorldExplorer
\end{itemize}

\textbf{Text2Room}
\begin{center}
	\includegraphics[width=\columnwidth]{images/text2room.jpeg}
	\label{fig:text2room}
\end{center}

\subsection{Camera-Controlled Video-to-3D}

\textbf{Explicit Camera Conditioning}
\begin{itemize}
	\item Pixel-wise raymaps or Plücker coordinates.
	\item Enables navigation and consistent viewpoint synthesis.
\end{itemize}

\begin{center}
	\includegraphics[width=\columnwidth]{images/cameractrl.jpeg}
	\label{fig:cameractrl}
\end{center}

\textbf{Key Methods}
\begin{itemize}
	\item CameraCtrl
	\item ViewCrafter
	\item GEN3C
\end{itemize}

\subsection{Discussion: Diffusion vs GANs}

\begin{itemize}
	\item Diffusion models require same input-output dimensionality.
	\item Advantage on GANs: discrimminate on images where no 3D representation is available.
	\item GANs allow partial or indirect supervision.
	\item Diffusion models can not be trained on partial data, only optimized (e.g. use 2D Loss to push 3D diffusion: Distillation of pre-trained model)
	\item For both: Performance ultimately limited by training data quality and diversity.
\end{itemize}
