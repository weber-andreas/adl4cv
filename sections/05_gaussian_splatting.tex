\section{Gaussian Splatting}
\subsection{Point-Based Splatting Surfels}
Belong to traditional point-based graphics methods.

\textbf{Definition}
Surfels (surface elements) are 3D points that approximate local surface patches.  
\begin{itemize}
    \item position $(x, y, z)$,
    \item a surface normal,
    \item a radius for disc size,
    \item color and optional confidence.
\end{itemize}


\textbf{Rendering Principle:}
\begin{enumerate}
    \item Project each surfel center into the camera.
    \item Render its elliptical footprint.
    \item Blend footprints in depth order (front to back): Gaussian kernel.
    \item Moving the camera closer scales the discs accordingly.
\end{enumerate}

\textbf{Advantages:}
\begin{itemize}
    \item No mesh connectivity required.
    \item Robust to noisy or incomplete geometry.
    \item Projection step of each point into image can be parallelized.
    \item (Almost) real-time rendering feasible.
\end{itemize}

\textbf{Limitations:}
\begin{itemize}
    \item Limited handling of transparency.
    \item Not volumetric; represents only surfaces.
    \item Blending requires heuristics and is not fully differentiable.
\end{itemize}

\subsection{Gaussian Splatting}

\textbf{3D Gaussian Splatting} represents a scene with anisotropic 3D Gaussian ellipsoids.  
Each Gaussian is defined by:
\begin{itemize}
    \item mean (3D center),
    \item covariance matrix (shape, scale, and orientation),
    \item color (often spherical harmonics),
    \item opacity.
\end{itemize}

\textbf{Spherical Harmonics}:
Are a family of smooth, wave-like functions defined on the surface of a sphere.

\textbf{Overview}
\begin{center}
    \includegraphics[width=\columnwidth]{images/gaussian_splatting.jpeg}
    \label{fig:gaussian_splatting}
\end{center}

\begin{itemize}
    \item Initialization: sparse structure from motion points are converted into 3D Gaussians
    \item Rendering Loop: \begin{itemize}
        \item The camera defines the viewpoint for rendering
        \item 3D Gaussians are projected into 2D image space
    \end{itemize}
    \item Differentiable Tile Rasterer: \begin{itemize}
        \item Sort Gaussians by depth
        \item Split screen into tiles for efficient rendering
        \item Alpha-blending: Blend Gaussians front-to-back using their opacities
        \item Fully differentiable: allows gradient-based optimization
    \end{itemize}
    \item Optimization: \begin{itemize}
        \item There is no neural network involved, optimization is performed directly on Gaussian parameters.
        \item Compare rendered image to ground truth: Photometric Loss
        \item Multi-view constraint: use multiple images from different viewpoints
        \item Backpropagate errors to update Gaussian parameters (position, shape, color, opacity)
        \item To get a valid covariance matrix, optimize its decomposition $\Sigma = RSS^TR^T$
    \end{itemize}
    \item Adaptive Density Control: \begin{itemize}
        \item Add new Gaussians in high-error regions
        \item Remove Gaussians in low-importance areas
        \item Split large Gaussians in high-variance regions
        \item Clone Gaussians in underfilled regions
    \end{itemize}
\end{itemize}

\subsubsection{Rendering Principle}
\begin{enumerate}
    \item \textbf{Sort Gaussians}: Globally based on depth from camera.
    \item \textbf{Splat}: Compute the shapes of the Gaussian after projection into 2D.
    \item \textbf{Blend}: Perform front-to-back alpha composite.
\end{enumerate}

\textbf{Advantages:}
\begin{itemize}
    \item NeRF-level quality with real-time rendering.
    \item Fully differentiable projection and blending.
    \item Supports volumetric effects (soft edges, transparency).
    \item Can be optimized end-to-end from images.
\end{itemize}

\textbf{Limitations:}
\begin{itemize}
    \item No explicit surface mesh.
    \item Memory usage can grow with dense scenes.
    \item Topology changes are harder to capture.
\end{itemize}

\subsection{Dynamic Scene Representation}
Make Gaussians move over time:
\textbf{Fixed Parameters}:
\begin{itemize}
    \item Scale
    \item Color
    \item Opacity
\end{itemize}
These get optimized on the first frame and after that stay constant.
\textbf{Time-Varying Parameters}:
\begin{itemize}
    \item 3D Position
    \item 3D Rotation
\end{itemize}
Optimize for each timestep relative to the previous.

\subsection{Surfels vs Gaussians}

\textbf{Surfels}: represent surface discs;  
\textbf{Gaussians}: represent smooth volumetric blobs.

\begin{itemize}
    \item Surfels are 2D oriented patches.
    \item Gaussians are 3D ellipsoids that naturally project to 2D.
    \item Surfels require heuristic blending; Gaussians use volumetric accumulation.
    \item Gaussians allow end-to-end differentiable optimization.
\end{itemize}

\begin{center}
    \includegraphics[width=\columnwidth]{images/surface_vs_gaussian.jpeg}
    \label{fig:surface_vs_gaussian}
\end{center}

\textbf{Surface Splatting:}

\[
c = \frac{\sum_{i \in \mathcal{N}} c_i w_i}{\sum_{i \in \mathcal{N}} w_i}
\]
Weighted average of colors $c_i$ with weights $w_i$. Simple normalization, no opacity.

\textbf{Volume Splatting (Alpha Compositing):}
\[
c = \sum_{i \in \mathcal{N}} c_i \alpha_i \prod_{j=1}^{i-1} (1 - \alpha_j w_j)
\]
\begin{itemize}
    \item Front-to-back blending where $\alpha_i$ is opacity of splat $i$. 
    \item The product term $\prod_{j=1}^{i-1} (1 - \alpha_j w_j)$ is the transmittance - how much light passes through all previous splats.\\
          Similar to NeRF volume rendering equation.
    \item Each splat contributes its color $c_i$ scaled by its opacity $\alpha_i$ and the accumulated transparency from splats in front.
\end{itemize}

 

\subsection{Summary}
Gaussian Splatting can be seen as a hybrid between Neural Radiance Fields and 3D Meshes
\begin{center}
    \includegraphics[width=\columnwidth]{images/gaussian_splatting_hybrid.jpeg}
    \label{fig:gaussian_splatting_hybrid}
\end{center}
\textbf{Fundamental Works}\\
\href{https://www.cs.umd.edu/~zwicker/publications/SurfaceSplatting-SIG01.pdf}{Surface Splatting, Zwicker, 2001}
\href{https://arxiv.org/pdf/2308.04079}{3D Gaussian Splatting for Real-Time Radiance Field Rendering, Kerbl, 2023}
